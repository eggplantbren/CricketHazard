\documentclass[letterpaper, 11pt]{article}
\usepackage{graphicx}
\usepackage{natbib}
\usepackage[left=3cm,top=3cm,right=3cm]{geometry}

\renewcommand{\topfraction}{0.85}
\renewcommand{\textfraction}{0.1}
\parindent=0cm

\title{Notes on Cricket Hazard Functions}
\author{Brendon J. Brewer}

\begin{document}
\maketitle

\section{Introduction}
For the basic idea and an initial study, see \citet{2008arXiv0801.4408B}.
For this software, I am using a different hazard function with more
easily interpreted parameters.

Consider a score $X \in \{0, 1, 2, ... \}$. I will now define various
quantities:

The probability distribution:
\begin{eqnarray}
f(x) = P(X = x)
\end{eqnarray}
The cumulative probability distribution (standard definition):
\begin{eqnarray}
F(x) = P(X \leq x) = \sum_{i=0}^x f(x)
\end{eqnarray}
The cumulative probability distribution (reversed definition):
\begin{eqnarray}
G(x) = P(X \geq x) = \sum_{i=x}^\infty f(x)
\end{eqnarray}
This reversed definition is great because it gives the likelihood function
for not out innings.
The Hazard function:
\begin{eqnarray}
H(x) = P(X = x | X \geq x) = \frac{f(x)}{G(x)}
\end{eqnarray}
The effective average:
\begin{eqnarray}
\mu(x) = \frac{1}{H(x)} - 1
\end{eqnarray}

\section{Model for Getting Your Eye In}
The effective average parameterisation. This is different from the original
paper. Parameters are $\mu_0$ (how good is the player when they first come
in?), $\mu_1$ (how good are they when they're warmed up?),
and $L$ (how long does it take (in terms of runs) for them to warm up?)
\begin{eqnarray}
\mu(x; \mu_0, \mu_1) = \mu_1 + \left(\mu_0 - \mu_1\right)\exp\left(-\frac{x}{L}\right)
\end{eqnarray}





\begin{thebibliography}{99}
\bibitem[Brewer(2008)]{2008arXiv0801.4408B} Brewer, B.~J.\ 2008.\ Getting 
Your Eye In: A Bayesian Analysis of Early Dismissals in Cricket.\ ArXiv 
e-prints arXiv:0801.4408.
\end{thebibliography}

\end{document}

