\documentclass[letterpaper, 11pt]{article}
\usepackage{graphicx}
\usepackage{natbib}
\usepackage[left=3cm,top=3cm,right=3cm]{geometry}

\renewcommand{\topfraction}{0.85}
\renewcommand{\textfraction}{0.1}
\parindent=0cm


% Stolen from http://stackoverflow.com/questions/586572/make-code-in-latex-look-nice
\usepackage{color}
\usepackage{listings}
\lstset{ %
language=Python,                % choose the language of the code
basicstyle=\footnotesize,       % the size of the fonts that are used for the code
numbers=left,                   % where to put the line-numbers
numberstyle=\footnotesize,      % the size of the fonts that are used for the line-numbers
stepnumber=1,                   % the step between two line-numbers. If it is 1 each line will be numbered
numbersep=5pt,                  % how far the line-numbers are from the code
backgroundcolor=\color{white},  % choose the background color. You must add \usepackage{color}
showspaces=false,               % show spaces adding particular underscores
showstringspaces=false,         % underline spaces within strings
showtabs=false,                 % show tabs within strings adding particular underscores
frame=single,           % adds a frame around the code
tabsize=2,          % sets default tabsize to 2 spaces
captionpos=b,           % sets the caption-position to bottom
breaklines=true,        % sets automatic line breaking
breakatwhitespace=false,    % sets if automatic breaks should only happen at whitespace
escapeinside={\%*}{*)}          % if you want to add a comment within your code
}


\title{Notes on Cricket Hazard Functions}
\author{Brendon J. Brewer}

\begin{document}
\maketitle

\section{Introduction}
For the basic idea and an initial study, see \citet{2008arXiv0801.4408B}.
For this software, I am using a different hazard function with more
easily interpreted parameters.

Consider a score $X \in \{0, 1, 2, ... \}$. I will now define various
quantities:

The probability distribution:
\begin{eqnarray}
f(x) = P(X = x)
\end{eqnarray}
The cumulative probability distribution (standard definition):
\begin{eqnarray}
F(x) = P(X \leq x) = \sum_{i=0}^x f(x)
\end{eqnarray}
The cumulative probability distribution (reversed definition):
\begin{eqnarray}
G(x) = P(X \geq x)
\end{eqnarray}
This reversed definition is great because it gives the likelihood function
for not out innings.
The Hazard function:
\begin{eqnarray}
H(x) = P(X = x | X \geq x) = \frac{f(x)}{G(x)}
\end{eqnarray}

\section{Construction from Effective Average}
The effective average measures how good a player is as a function
of their current score. It is related to the hazard function:
\begin{eqnarray}
\mu(x) = \frac{1}{H(x)} - 1
\end{eqnarray}

If $\mu(x)$ is specified and we want to construct everything else,
the formulae are:

\begin{eqnarray}
H(x) = \frac{1}{\mu(x) + 1}
\end{eqnarray}

\begin{eqnarray}
G(x) &=& G(x-1)\left(1 - H(x-1)\right) \\
&=& G(x-1)\left(\frac{\mu(x-1)}{\mu(x-1) + 1}\right)
\end{eqnarray}
with initial condition $G(0) = 1$. In the code, this is implemented
by:
\begin{lstlisting}
# Effective average
self.mu = self.mu1 + (self.mu0 - self.mu1)*\
np.exp(-HazardModel.x/self.L)
# Log of reversed cumulative distribution
self.logG = np.zeros(HazardModel.x.shape)
self.logG[1:] = np.cumsum(np.log(self.mu[0:-1])\
		 - np.log(self.mu[0:-1] + 1.0))
# Log of probability distribution
self.logf = self.logG - np.log(self.mu + 1.0)
\end{lstlisting}

\begin{eqnarray}
G(x) = G(x-1)\left(1 - H(x-1)\right)
\end{eqnarray}


\section{Model for Getting Your Eye In}
The effective average parameterisation. This is different from the original
paper. Parameters are $\mu_0$ (how good is the player when they first come
in?), $\mu_1$ (how good are they when they're warmed up?),
and $L$ (how long does it take (in terms of runs) for them to warm up?)
\begin{eqnarray}
\mu(x; \mu_0, \mu_1) = \mu_1 + \left(\mu_0 - \mu_1\right)\exp\left(-\frac{x}{L}\right)
\end{eqnarray}





\begin{thebibliography}{99}
\bibitem[Brewer(2008)]{2008arXiv0801.4408B} Brewer, B.~J.\ 2008.\ Getting 
Your Eye In: A Bayesian Analysis of Early Dismissals in Cricket.\ ArXiv 
e-prints arXiv:0801.4408.
\end{thebibliography}

\end{document}

